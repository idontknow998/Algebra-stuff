\documentclass{article}
\usepackage{amsmath}
\usepackage{amsfonts}
\usepackage{amssymb}
\title{Algebra HW 2}
\author{Michael Lange}
\begin{document}
\section{Cyclic Groups}
\paragraph{4.5}
Suppose that $G$ is a cyclic group. Let $H\subset G$ be a subgroup. There exists an element $x\in G$ such that every element of $G$ is of the form $x^k$ for some $k\in \mathbb{Z}$. Then every element of $H$ is of the form $x^k$ for some $k\in \mathbb{Z}$. Define $M=\{m\in \mathbb{Z} :x^m\in H\}$. Since $H$ is a subgroup of $G$, the identity element of $G$, 1, is in $H$. $1=x^0$ so $0\in M$. If $k\in M$, $x^k\in H$ so $(x^k)^{-1}=x^{-k}\in H$. Therefore, $-k\in M$. If $k,l\in M$ then $x^k, x^l\in H$ so $x^k*x^l=x^{k+l}\in H$. Therefore, $k+l\in M$. Thus, $M$ is a subgroup of $\mathbb{Z}$, so $M=\mathbb{Z}c$ where $c$ is the smallest positive element of $M$. Every element of $H$ is of the form $x^m$ for some $m\in M$, but every $m\in M$ is $kc$ for some $k\in \mathbb{Z}$, so every element of $H$ is $x^{kc}=(x^c)^k$ for some $k\in \mathbb{Z}$. Therefore $H$ is a cyclic group generated by $x^c$.

\paragraph{4.6}
\subparagraph{(a)}
Let $G$ be a cyclic group generated by an element $x$. If $G$ is of order 6, it is generated by 2 of its elements, namely $x$ and $x^5$. If $G$ is of order 5, it is generated by all 4 of its non-identity elements. If $G$ is of order 8, it is generated by 4 of its elements, namely $x, x^3, x^5$ and $x^7$. The proof follows:
\subparagraph{(b)}
Let $G$ be a cyclic group of finite order $n$. Then the number of elements of $G$ that generate $G$ is the number of integers less than and relatively prime to $n$. More specifically, if $G$ is generated by an element $x$, then for $k<n$, it is also generated by $k$ iff $k$ is relatively prime to $n$. For suppose that $k<n$ is not relatively prime to $n$. Then there exists $c\in \mathbb{Z}, c>1$ such that $c|k$ and $c|n$. Then for some $n_0, k_0\in \mathbb{Z}, n=cn_0$ and $k=ck_0$. Then $(x^k)^{n_0}=x^{kn_0}=x^{ck_0n_0}=x^{nk_0}=(x^n)^{k_0}=1$. Thus $x^k$ has order at most $n_0$, so the subgroup generated by $x^k$ has at most $n_0$ elements. Since $c>1$, $n_0<n$, so this subgroup has fewer than $n$ elements and cannot be all of $G$.\newline
Now suppose that $k<n$ is relatively prime to $n$. Let $c\in \mathbb{Z}$ be the order of $x^k$. Then $1\le c\le n$, and $x^{kc}$=1. Then, $n|kc$. Since $n$ and $k$ are relatively prime, this implies that $n|c$. But $1\le c\le n$ so $c=n$. Therefore, $x^k$ has order $n$ and the subgroup of $G$ that it generates has $n$ elements. Since $G$ has only $n$ elements, this subgroup is the entirety of $G$.

\paragraph{4.7)}
Let $G$ be a group and $x,y$ elements of $G$ such that $x, y, xy$ are all over order 2. Let $H=\{1,x,y,xy\}$. The following computations show that $H$ is closed under the operation: $1z=z1=z\in H$ for any $z\in H$. Any element multiplied with itself is the identity and thus in $H$. As for $xy$, since $xyxy=1$, $x^2yxy=x$, and $yxy=x$ since $x^2=1$. Then $yxy^2=xy$, so $yx=xy$ since $y^2=1$. Thus $yx\in H$. The remaining products are $x^2y,xy^2,xyx,$ and $yxy$. Since $x$ and $y$ commute as shown, these reduce to just the two products $x^2y$ and $xy^2$, which are $y$ and $x$ respectively. Therefore $H$ is closed under the operation. Furthermore, $H$ contains an identity element, $1$. Lastly, since every element is its own inverse, $H$ contains inverse and is therefore a group.\newline
To verify that $H$ has order 4, its elements must be proven distinct. First, since $x,y,xy$ are all of order 2, none can equal $1$ which has order $1$. Then neither can $x=xy$ nor $y=xy$ since by cancellation these would yield $1=y$ and $1=x$, respectively. Lastly, if $x$ were equal to $y$, $xy$ would equal $1$ since $x^2=1$. Therefore $x\ne y$. All four elements of $H$  as listed are in fact distinct, so $H$ has order 4.

\paragraph{4.8)}

\paragraph{4.9)}
Every element of $S_4$ can be decomposed into disjoint cycles. If a permutation $p$ is written as a product of disjoint cycles, these cycles commute, so the $n$th power of $p$ is the product of the $n$th powers of these disjoint cycles. Furthermore, the order of a cycle is its length. An element of $S_4$ is either a 4-cycle, a 3-cycle, or a product of (1 or 2) disjoint  2-cycles. A 4- or 3-cycle does not have order 2 since it has order 4 or 3. Taking the second power of a product of 2-cycles yields the identity, since each individual 2-cycle to the second power yields identity. There are 9 distinct elements of $S_n$ that are a product of (1 or 2) disjoint 2-cycles: $(12),(13),(14),(23),(24),(34),(12)(34),(13)(24),(14)(23)$. Therefore $S_4$ contains 9 elements of order 2.

\paragraph{4.10)}

If $G$ is abelian, let $a,b\in G$ be elements with finite orders $m,n$ respectively. By commutativity, $(ab)^mn=a^mnb^mn$=1, so $ab$ must have finite order.

\section{Homomorphisms}

\paragraph{5.1)}
Let $\varphi :G\to G'$ be a surjective homomorphism. Suppose $G$ is cyclic. Then there exists $x\in G$ such that every element of $G$ is $x^k$ for some $k\in \mathbb{Z}$. Since $\varphi$ is surjective, for any element $g'\in G'$, there exists $g\in G$ such that $\varphi (g)=g'$. Then there exists $k\in \mathbb{Z}$ such that $\varphi (x^k)=g'$. Then, $\varphi (x)^k=g'$ Thus, $\varphi (x)$ generates $G'$, so $G'$ is cyclic.\newline
Suppose that $G$ is abelian. For any $g'_1,g'_2\in G'$, there exist $g_1,g_2\in G$ such that $\varphi (g_1)=g'_1, \varphi (g_2)=g'_2$. Then, $g'_1g'_2=\varphi (g_1)\varphi (g_2)=\varphi (g_1g_2)=\varphi(g_2g_1)=\varphi (g_2)\varphi (g_1)=g'_2g'_1$. Thus, $G'$ is abelian.

\paragraph{5.2)}
Let $G$ be a group and $H,K$ subgroups of $G$. Consider $H\cap K$. If $a,b\in H\cap K$, then $a,b\in H$ and $a,b\in K$. Therefore, $ab\in H$ and $ab\in K$. Thus, $ab\in H\cap K$. Since $1\in H$ and $1\in K$, $1\in H\cap K$. Lastly, if $a\in H\cap K$, then $a\in H$ and $a\in K$ so $a^{-1}\in H$ and $a^{-1}\in K$, so $a^{-1}\in H\cap K$. Therefore, $H\cap K$ is a group, and since it is a subset of $H$, it is a subgroup of $H$.\newline
Now suppose that $K$ is a normal subgroup of $G$. Consider an arbitrary $x\in H\cap K$ and an arbitrary $h\in H$. Since $h\in G$ and $K$ is a normal subgroup of $G$, $hxh^{-1}\in K$. Since $x,h,h^{-1}$ are all elements of $H$,  $hxh^{-1}\in H$, so $hxh^{-1}\in H\cap K$. Therefore, $H\cap K$ is a normal subgroup of $H$.

\end{document}